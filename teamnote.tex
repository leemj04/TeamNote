
\documentclass[9pt,landscape,a4paper,twocolumn]{extarticle}

\setlength{\columnsep}{5pt}

\usepackage[left=0.2cm, right=0.2cm, top=1cm, bottom=1.2cm, headsep=0.2cm]{geometry}
\usepackage{amsmath}
\usepackage{amssymb}
\usepackage{fontspec}
\usepackage{kotex}
\usepackage{graphicx}
\usepackage{setspace}
\usepackage{listings}
\usepackage{comment}
\usepackage{import}
\usepackage{wrapfig}
\usepackage{url}
\usepackage{array}
\usepackage{titlesec}
\usepackage[normal]{engord}
\usepackage[svgnames,table]{xcolor}

\usepackage{fancyhdr}
\pagestyle{fancy}
\fancyhead[R]{\thepage}
\fancyhead[L]{KU - unipass}

\setmonofont[
    BoldFont = consolab.ttf,
    ItalicFont = consolai.ttf
]{consola.ttf}
\setmainhangulfont{NanumMyeongjo}
\setlength\parindent{0pt}
\usepackage[parfill]{parskip}
\setlength\parskip{0pt}
\setlength\baselineskip{0pt}
\titlespacing*{\section}{0pt}{0pt}{0pt}
\titlespacing*{\subsection}{0pt}{0pt}{0pt}

\definecolor{dkgrey}{RGB}{127, 127, 127}

\lstset{
    basicstyle=\footnotesize\ttfamily,
    breaklines=true,
    breakindent=1.1em
%    numbers=left,
%    numberstyle=\footnotesize\ttfamily\color{dkgrey},
%    numbersep=5pt
%    frame=trbl
}

\lstdefinestyle{mycpp}{
  language=[GNU]C++,
  keywordstyle=\color{blue},
  commentstyle=\itshape\color{purple!40!black},
  stringstyle=\color{orange},
}
\lstdefinestyle{myjava}{
  language=Java,
  keywordstyle=\color{blue},
  commentstyle=\itshape\color{purple!40!black},
  stringstyle=\color{orange},
}

\begin{document}
\tableofcontents


\section{Setting}

\subsection{Default code}
\lstinputlisting[style=mycpp]{src/setting/default-code.cpp}

\section{Math}

\subsection{Extended Euclidean Algorithm}
\lstinputlisting[style=mycpp]{src/math/extended-euclidean.cpp}

\subsection{Primality Test}
\lstinputlisting[style=mycpp]{src/math/primality-test.cpp}

\subsection{Integer Factorization (Pollard's rho)}
\lstinputlisting[style=mycpp]{src/math/pollard-rho.cpp}

\subsection{Chinese Remainder Theorem}
\lstinputlisting[style=mycpp]{src/math/crt.cpp}

\subsection{Query of nCr mod M in $O(Q+M)$}
\lstinputlisting[style=mycpp]{src/math/binomial.cpp}

\subsection{pelindrome number}
\lstinputlisting[style=mycpp]{src/math/pelindrome.cpp}

\subsection{Matrix Pow}
\lstinputlisting[style=mycpp]{src/math/matrix-mul-pow.cpp}

\subsection{Catalan, Derangement, Partition, 2nd Stirling}
$C_n = \frac{1}{n + 1} \binom{2n}{n}, C_0 = 1, C_{n + 1} = \sum_{i=0}^{n} C_i C_{n-i},  C_{n + 1} = \frac{2(2n+1)}{n+2} C_n$

$D_n = (n-1)(D_{n-1} + D_{n-2})=n! \sum_{i=1}^n \frac{(-1)^{i+1}}{i!}$

$ P(n) =\sum_{k \in \mathbb{Z}\setminus\{0\}}^{}
    (-1)^{k+1} P(n-k(3k-1)/2) $

$= P(n-1) + P(n-2)-P(n-5)-P(n-7) +P(n-12) +P(n-15) - P(n-22) -\cdots$

$P(n,k)=P(n-1,k-1)+P(n-k,k)$, $S(n,k)=S(n-1,k-1)+k\cdot S(n-1,k)$

\subsection{Matrix Operations}
\lstinputlisting[style=mycpp]{src/math/matrix-operations.cpp}

\subsection{Gaussian Elimination}
\lstinputlisting[style=mycpp]{src/math/gaussian.cpp}

\subsection{Permutation and Combination}
\lstinputlisting[style=mycpp]{src/math/permutation.cpp}
\lstinputlisting[style=mycpp]{src/math/combination.cpp}

\subsection{Lifting The Exponent}
For any integers $x$, $y$ a positive integer $n$, and a prime number $p$ such that $p \nmid x$ and $p \nmid y$, the following statements hold:

\begin{itemize}
    \item When $p$ is odd:
    \begin{itemize}
        \item If $p \mid x-y$, then $\nu_p(x^n-y^n) = \nu_p(x-y)+\nu_p(n)$.
        \item If $n$ is odd and $p \mid x+y$, then $\nu_p(x^n+y^n) = \nu_p(x+y)+\nu_p(n)$.
    \end{itemize}
    \item When $p = 2$:
    \begin{itemize}
        \item If $2 \mid x-y$ and $n$ is even, then $\nu_2(x^n-y^n) = \nu_2(x-y)+\nu_2(x+y)+\nu_2(n)-1$.
        \item If $2 \mid x-y$ and $n$ is odd, then $\nu_2(x^n-y^n) = \nu_2(x-y)$.
        \item Corollary:
        \begin{itemize}
            \item If $4 \mid x-y$, then $\nu_2(x+y)=1$ and thus $\nu_2(x^n-y^n) = \nu_2(x-y)+\nu_2(n)$.
        \end{itemize}
    \end{itemize}
    \item For all $p$:
    \begin{itemize}
        \item If $\gcd(n,p) = 1$ and $p \mid x-y$, then $\nu_p(x^n-y^n) = \nu_p(x-y)$.
        \item If $\gcd(n,p) = 1$, $p \mid x+y$ and $n$ odd, then $\nu_p(x^n+y^n) = \nu_p(x+y)$.
    \end{itemize}
\end{itemize}

\section{Data Structure}
\subsection{Lazy Segment Tree}
\lstinputlisting[style=mycpp]{src/data-structure/lazy-segment-tree.cpp}

\subsection{Persistent Segment Tree}
\lstinputlisting[style=mycpp]{src/data-structure/persistent-segment-tree.cpp}

\subsection{Strongly Connected Component}
\lstinputlisting[style=mycpp]{src/data-structure/scc.cpp}
\section{DP}

\subsection{LIS}
\lstinputlisting[style=mycpp]{src/dp/lis.cpp}

\section{Graph}

\subsection{Dijkstra}
\lstinputlisting[style=mycpp]{src/graph/dijkstra.cpp}

\subsection{LCA}
\lstinputlisting[style=mycpp]{src/graph/lca.cpp}

\subsection{Centroid Decomposition}
\lstinputlisting[style=mycpp]{src/graph/centroid-decomposition.cpp}

\subsection{Minimum Spanning Tree}
\lstinputlisting[style=mycpp]{src/graph/mst.cpp}

\subsection{Offline Dynamic Connectivity}
\lstinputlisting[style=mycpp]{src/graph/offline-dynamic-connectivity.cpp}

\section{String}

\subsection{KMP}
\lstinputlisting[style=mycpp]{src/string/kmp.cpp}

\subsection{Z Algorithm}
\lstinputlisting[style=mycpp]{src/string/z.cpp}

\subsection{LCS}
\lstinputlisting[style=mycpp]{src/string/lcs.cpp}

\section{Geometry}

\subsection{CCW}
\lstinputlisting[style=mycpp]{src/geometry/ccw.cpp}

\section{Hash}

\subsection{Basic Hash}
\lstinputlisting[style=mycpp]{src/hash/basic-hash.cpp}

\end{document}
